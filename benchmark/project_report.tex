\documentclass{article}

% Packages
\usepackage[utf8]{inputenc}  % Encoding
\usepackage[T1]{fontenc}     % Font encoding
\usepackage{lmodern}         % Improved font rendering
\usepackage{amsmath}         % Math symbols
\usepackage{amsfonts}        % Math fonts
\usepackage{amssymb}         % Additional math symbols
\usepackage{graphicx}        % Include graphics
\usepackage{hyperref}        % Hyperlinks
\usepackage{geometry}        % Page layout
\usepackage{listings}
\usepackage{physics}
\usepackage{subcaption}
% \geometry{a4paper, margin=1in}
% TODO add more graphs, check grammar
% Title, author, and date
\title{Numerically Solving Navier Stokes equation with GPUs}
\author{Gašper Golob}
\date{\today}

\begin{document}

% Title
\maketitle

% Introduction
\section{Introduction}
The goal of the project is to use the GPU to accelerate the execution of two methods for 
solving the Navier Stokes equation. Specifically the two methods that are being accelerated are
the pressure projection and artificial compressibility method.

The performance and accuracy of the GPU based versions are then tested on the 
lid driven cavity problem. The lid driven cavity problem is the case where the fluid is in 
a square case, where one of the sides induces some movement. 
The case is schematically presented in the figure \ref{fig:lid_driven_cavity}, 
from \cite{lidDriven} alongside an example final velocity field.
\begin{figure}[h!] 
    \centering
    \begin{subfigure}[t]{0.4\textwidth}
    \centering 
    \includegraphics[width=1\textwidth]{plots/lid_driven_cavity.png} 
    \caption{The schematic of the lid driven cavity problem.} 
    \end{subfigure}
    \begin{subfigure}[t]{0.4\textwidth}
    \centering 
    \includegraphics[width=1\textwidth]{plots/lidDriven_problem.png} 
    \caption{An example of the final velocity field, at the end of a 50 second simulation 
    using the pressure projection method.} 
    \end{subfigure}
    \caption{Schematic and example velocity field of the lid driven cavity problem.}
    \label{fig:lid_driven_cavity} 
\end{figure}
It is often used to test numerical methods for fluid simulations as the velocities converge towards 
some value, which can tell us if implemented method is stable.

The methods work iteratively, where at each time step, they compute new velocities using the 
Navier Stokes equation. Afterwards they correct the pressure using an assumed pressure volume 
invariant. 

For the purposes of this project the details of solving the partial differential equations are not 
as important, since the project is mostly concerned with accelerating the solving of systems and 
various matrix operations.

The GPU versions are also compared to their base versions to gauge the level of improvement and to 
see if they remain accurate.

The project is made using C++, where the discretization of the differential equations is done using 
the medusa library \cite{medusa}. To solve linear systems and work with matrices on the CPU the 
Eigen library is used. Similarly, to work with matrices and vectors on the GPU side, 
CUDA libraries such as cuSolver, cuSparse, cuDSS and Thrust are used.
\section{Pressure projection method}
The pressure correction method explicitly steps through time using the equation 
\begin{equation}
    \frac{\rho}{\Delta t} \left( \vec{v}(t + \Delta t) - \vec{v}(t)
    \right) =
    -\grad p + \mathcal{F}(\vec{v}(t)),
\end{equation}
where \(t,~\Delta t,~\vec{v},~p,~\rho,~\mathcal{F}\) represent the time, time step, velocity 
field, pressure field, density and all the non-pressure forces. 
Every time step is split into two parts, the first gives us an intermediate velocity
which does not consider the pressure gradient
\begin{equation}
    \label{eq:intermediate_step}
    \vec{v}^* = \vec{v}(t) + \frac{\Delta t}{\rho}
        \mathcal{F}(\vec{v}(t)),
\end{equation}
while the second applies the pressure correction to the intermediate step
\begin{equation}
    \vec{v}(t + \Delta t) = \vec{v}^* -\frac{\Delta t}{\rho} \grad p.
\end{equation}
To compute the pressure correction we use the equality
\begin{equation}
    \label{eq:pressure_correction}
    \div \vec{v}^*
        = \frac{\Delta t}{\rho}\laplacian p.
\end{equation}
The equations are solved with the help of matrices generated using the medusa library.
For the purposes of this project we are mostly concerned with solving the linear 
systems belonging to the equations \ref{eq:intermediate_step} and \ref{eq:pressure_correction}.

For equation \ref{eq:intermediate_step} the matrix changes at each time step, so we need 
a there isn't any structure we could exploit to speed up the execution. On the other hand 
the matrix belonging to the equation \ref{eq:pressure_correction} remains fixed, so 
we need a system solver that leverages this. This is especially important, since 
the pressure correction can be executed up to 100 times per time step.


\subsection{Solvers}
All solvers are initialized using the Eigen \verb|SparseMatrix| type for sparse matrices. 
To solve a system they receive the right-hand side vector which has the Eigen type \verb|VectorXd|,
which is a dense vector whose elements have the type \verb|double|. Ultimately they return 
a vector of the same type.
\subsubsection{SparseLU solver}
The base CPU version uses the Eigen library to solve the linear systems. For both of the systems 
the SparseLU solver is used, which implements the supernodal LU factorization of the matrix, and then
solves the system using the computed triangular matrices.
\subsubsection{QR solver}
The QR solver uses the function \verb|cusolverSpDcsrlsvqr| from the cuSolverSp library. 
The function takes as arguments a sparse matrix \(A\) in the 
compressed sparse rows format alongside two dense vectors \(x\) and \(b\). It solves the 
system \(Ax=b\) on the GPU. In the background the method computes a sparse QR factorization of the matrix \(A\),
while also providing some reordering, to minimize zero fill in. For the purposes of this project,
the \verb|symrcm| reordering scheme is used.
\subsubsection{cuDSS solver}
The cuDSS solver is the main solver currently implemented in the cuDSS library. Since the matrix 
doesn't have any extra properties it uses the LDU decomposition. Same as the previous 
solver the matrix is in the CSR format, while the right-hand side and solution vectors
are dense. All the matrices and vectors are transferred to the GPU where the decomposition is 
computed and the systems are solved.
\subsubsection{RF solver}
The refactorization solver is meant to sequentially solve multiple systems of the form 
\(A x_i = b_i\) for multiple vectors \(x_i\), \(b_i\) and a fixed matrix \(A\). 

The first time 
it has to solve a system it uses the \verb|cusolverSP_lowlevel_preview| library. This library 
first computes the sparse LU decomposition (with some reordering) of the matrix and then solves the 
first system.

Afterwards the sparse LU  decomposition is extracted and passed to the \verb|cuSolverRF| library,
which can then use it to solve later systems.

The first iteration is done completely on the CPU, as the \verb|cusolverSP_lowlevel_preview| library
currently lacks GPU versions of the functions. Later, the systems are solved using the functions 
from \verb|cuSolverRF| which works on the GPU.

\subsection{Benchmarks}
To speed up the solving of the system belonging to equation \ref{eq:intermediate_step} we use 
the QR and cuDSS solvers, while
the second system is only accelerated using the RF solver, as this is the only solver that allows 
for preprocessing. 
The execution time of the programs, when using different solvers can be seen in the
figure \ref{fig:lidDriven_time}.

We can see that the lid driven cavity version which is the reference CPU implementation 
is the fastest at the start, when the matrices are smaller and the CPU's better single 
thread performance results in better overall performance.

Afterwards, the solver that uses the cuDSS library for the system of the equation \ref{eq:intermediate_step} 
becomes the fastest, with the difference increasing for larger \(N\). The cuDSS library has also been 
created the most recently, so it might also be leveraging more modern GPU features than the other versions.

The worst case is when we use the QR solver to solve the system of \ref{eq:intermediate_step}.
This could be due to the fact that it uses a sparse QR decomposition instead of a sparse LU decomposition
and the structure of the matrix does not lend itself to efficient QR decompositions.

The last case uses the cuDSS solver for the system belonging to \ref{eq:intermediate_step} 
and the RF solver for the system belonging to \ref{eq:pressure_correction}. 
Unfortunately it seems that solving the LU system on the GPU slows down the execution, 
which makes some sense as solving triangular systems is a very sequential operation.

Regardless, it might prove useful to run the benchmarks on even larger cases, as there might be some 
performance improvements for very large \(N\). The limiting factor here might however be, that 
the LU decomposition is done on the CPU, so the setup time might take too long.
\begin{figure}[h!] 
    \centering 
    \includegraphics[width=0.8\textwidth]{plots/lidDriven_time.png} 
    \caption{Graph of times to simulate 50 seconds of the lid driven cavity problem using
    the pressure projection method, for different problem sizes.} 
    \label{fig:lidDriven_time} 
\end{figure}

From the figure \ref{fig:lidDriven_comp} we can see that for all methods the slowest part is 
the solving of the system for \ref{eq:intermediate_step}. In the case of the RF solver we can also 
clearly see that the RF solver slows down the execution. We can also conclude that focusing on 
speeding up the execution of the system of \ref{eq:intermediate_step} makes the most sense,
as other parts are relatively faster.
\begin{figure}[h!] 
    \centering 
    \includegraphics[width=\textwidth]{plots/lidDriven_comp.png} 
    \caption{Comparison of how much time different parts of the program take for different methods.} 
    \label{fig:lidDriven_comp} 
\end{figure}
From the figure \ref{fig:lidDriven_convergence}, for the specific case where the matrices 
are of size \(N = 6208\), we can see that the 
method converges toward some fixed value, which is what we expected. 
We can also check that for the case of \(N = 4570\) the magnitude of velocities at the middle cross-section 
match between all implementations. To see that the methods converge to the same final value of 
max \(u_y\) we can also look at \ref{fig:lidDriven_conv}.
\begin{figure}[h!] 
    \centering 
    \includegraphics[width=0.8\textwidth]{plots/lidDriven_conv.png} 
    \caption{Graph of the final max $u_y$ with regard to the problem size using the pressure projection method.} 
    \label{fig:lidDriven_conv} 
\end{figure}
\begin{figure}[h!] 
    \centering 
    \includegraphics[width=0.8\textwidth]{plots/lidDriven_convergence.png} 
    \caption{Graph of max $u_y$ in regard to number of steps taken using the pressure projection method.} 
    \label{fig:lidDriven_convergence} 
\end{figure}
\begin{figure}[h!] 
    \centering 
    \includegraphics[width=0.8\textwidth]{plots/lidDriven_cross.png} 
    \caption{Graph of velocities at the middle line after 50 seconds using the pressure projection method.} 
    \label{fig:lidDriven_cross} 
\end{figure}
\section{Artificial compressibility method}
For the artificial compressibility method we use the explicit Euler method to solve 
\begin{equation}
    \vec{v}^* = \vec{v} + \Delta t \left( \nu \nabla^2 \vec{v} - \vec{v}
    \cdot \nabla\vec{v} + \vec{g} \right), 
\end{equation}
for the intermediate step, where \(\Delta t,~\vec{v},~\nu\) are the time step, velocity field 
and viscosity.

Afterwards we iteratively compute the new pressure and velocity using 
\begin{align}
    \vec{p} & \leftarrow \vec{p} - \Delta t C^2  \rho  (\nabla \cdot \vec{v}), \\
    \vec{v} &\leftarrow \vec{v}^* - \frac{\Delta t}{\rho} \nabla \vec{p},
\end{align}
where \(p\) is the pressure field and \(\rho\) is the density. The magnitude of the 
artificial compressibility is determined using 
\begin{equation}
    C = \beta \max(\max_i(\lVert\vec{v_i}\rVert_2),\lVert\vec{v}_{ref}\rVert_2), 
\end{equation}
where \(\beta\) is the compressibility of the fluid.

From these equations some matrices are constructed for which multiplication needs to be 
sped up. Additionally, some element-wise operations can also be done on the GPU to save time.
In fact, after the matrices are constructed, the rest of the work can be done on the GPU.
To achieve this we need the operations described in the next subsection.
\subsection{Basic GPU operations and types}
For all operations sparse matrices are given in the compressed sparse row format and the vectors
are the usual dense vectors given as arrays on the GPU.
\subsubsection{VectorGPU and MatrixGPU}
These are some basic wrapper classes that convert to and from their Eigen counterparts.
Data that is passed to them is transferred to the GPU, where it can be used with other functions.
\subsubsection{Matrix multiplication}
To multiply sparse matrices with vectors the cuSparse library is used. More specifically the 
\verb|cusparseSpMV| function is used which takes a matrix in the compressed sparse rows format, 
a dense vector and returns a dense solution vector. Some multiplication data is initialized 
before the simulation as it speeds up the later multiplications. 
\subsubsection{Transforms and reduces}
For element-wise operations on dense vectors thrust's \verb|transform| and \verb|transform_reduce| 
functions are used.

The first use case is to allow element-wise vector multiplication, for which we can use 
the \verb|transform| function and the built-in \verb|multiplies| operator.

The second use case is to define a new operator \verb|axpy_functor| which is initialized using some 
value \(\alpha\) and which then for given vectors \(x\) and \(y\) computes \(y = \alpha x + y\).
The operator is again applied using the \verb|transform| function.

Another case is the definition of the operator \verb|abs_functor| which for a given vector 
computes the element-wise absolute value.

The last two operators that are defined are \verb|u_tuple_functor| and \verb|tuple_max_functor|,
which are then used with the \verb|transform_reduce| function. The \verb|u_tuple_functor| is initialized
using an array which represents a vector field. Running the functor on some set of indices
will return an array of pairs of said indices and the corresponding second coordinate values.
Then the \verb|tuple_max_functor| finds the maximum value among the values which have corresponding 
indices. This is used to find 
the maximum velocity in the \(y\) direction at the middle line of the lid driven cavity case.

Another function from Thrust that is used is \verb|max_element| which simply returns the 
maximum element of an array.
\subsection{Benchmarks}
We use the previously described functions to port the CPU code to the GPU. Additionally,
we also use CUDA streams to execute multiple operations on the graphics card at once, where 
this is possible. 

We can see the execution times of the base CPU version in comparison the GPU version on figure 
\ref{fig:lidDrivenACM_time}.
Once again the CPU version is better at the smallest values of \(N\), but it quickly becomes 
better as \(N\) grows larger.
\begin{figure}[h!] 
    \centering 
    \includegraphics[width=0.8\textwidth]{plots/lidDrivenACM_time.png} 
    \caption{Graph of times to simulate 50 seconds of the lid driven cavity problem using
    the artificial compressibility method, for different problem sizes.} 
    \label{fig:lidDrivenACM_time} 
\end{figure}

Much like for the implicit method we can also observe from \ref{fig:lidDrivenACM_convergence} that 
the methods converge towards some fixed value. Their middle line velocities also match, which we can 
see from \ref{fig:lidDrivenACM_cross}.

\section{Further work}
While the project is nearing completion there is some more work to be done. For an instance it 
would be good to benchmark the performance for some even larger \(N\), to see if the trends continue
as the amount of data on the GPU increases. It might also make sense to do some more detailed 
benchmarks to see how fast particular parts of the code are.

Besides that, the repository still needs to be cleaned up, and the code should be made easier to run
as it is currently setup specifically for the server, which was used for the project's development. 
% References
\begin{thebibliography}{99}
    \bibitem{medusa} Jure Slak and Gregor Kosec. 2021. Medusa: A C++ Library
    for Solving PDEs Using Strong Form Mesh-free Methods. ACM Trans. Math. Softw. 47,
    3, Article 28 (September 2021), 25 pages. https://doi.org/10.1145/3450966

    \bibitem{lidDriven} G. Kosec; A local numerical solution of a fluid-flow problem
    on an irregular domain, Advances in engineering software, vol. 120, 2018
    DOI: 10.1016/j.advengsoft.2016.05.010
\end{thebibliography}

\begin{figure}[h!] 
    \centering 
    \includegraphics[width=0.8\textwidth]{plots/lidDrivenACM_convergence.png} 
    \caption{Graph of max $u_y$ in regard to number of steps taken using the explicit method.} 
    \label{fig:lidDrivenACM_convergence} 
\end{figure}
\begin{figure}[h!] 
    \centering 
    \includegraphics[width=0.8\textwidth]{plots/lidDrivenACM_cross.png} 
    \caption{Graph of velocities at the middle line after 50 seconds using the explicit method.} 
    \label{fig:lidDrivenACM_cross} 
\end{figure}
\end{document}
